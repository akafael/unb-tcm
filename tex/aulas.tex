% Pacotes e configurações padrão do estilo "article"\
% -------------------------------------
\documentclass[a4paper,twocolumn,11pt]{book} 
% Layout
% --------------------------------------------------------------------------------
\input{relat_layout.tex}
\usepackage{multicol}
\usepackage[makestderr]{pythontex}
\restartpythontexsession{\thesection}

\newcommand{\tituloRelatorio}{Transporte de Calor e Massa\\Notas de Aula}
\title{\tituloRelatorio}
\hypersetup{pdftitle={\tituloRelatorio}}% title

% Definições Auxiliares
% -----------------------------------------------------------------
%\input{relat_aux.tex}

% ----------------------------------~>ø<~---------------------------------------
\begin{document}
% Capa e Índice ---------------------------------------------------------------
\maketitle
\tableofcontents
% Conteudo -------------------------------------------------------------------
\chapter{Termodinâmica} 
\section{}
\subsection{Lei dos Gases Ideais}
\begin{sympycode}
# Symbolic
P = Symbol("P")
V = Symbol("V")
n = Symbol("n")
R = Symbol("R")
T = Symbol("T")

m = Symbol("m")
M = Symbol("M")

# Numeric
nR = 8.31
\end{sympycode}

\begin{equation}
\sympy{P*V} = \sympy{n*R*T}
\end{equation}

\begin{table}[H]
\begin{tabular}{r@{ : }l}
$P$ & Pressão $[N/m^2]$ \\
$V$ & Volume $[m^3]$ \\
$n$ & número de mols \\
$R$ & Constante Universal dos Gases \\
$P$ & Temperatura $[K]$ \\
\end{tabular}
\end{table}

$R = \sympy{nR} Nm/(mol K)$

\begin{equation}
\sympy{n} = \sympy{m/M}
\end{equation}

\begin{table}[H]
\begin{tabular}{r@{ : }l}
$m$ & massa $[Kg]$ \\
$M$ & massa molar $[Kg/mol]$ \\
\end{tabular}
\end{table}

\subsubsection{Ex1}
\subsection{Entalpia}
\paragraph{Entalpia} : Energia Interna somada à energia de fluxo. 

\begin{sympycode}
H = Symbol("H")
Cp = Symbol("C_p")
Cv = Symbol("C_v")
C = Symbol("C")

U = Symbol("U")
\end{sympycode}

\begin{equation}
\Delta H = \sympy{m*Cp}\Delta T
\end{equation}

Para sólidos e líquidos:

\paragraph{Gases:}
$$\sympy{H} = \sympy{U+ P*V}$$
$$(\sympy{m*Cp*T}) = (\sympy{m*Cv*T}) + (\sympy{n*R*T})$$
$$\sympy{Cp} = \sympy{Cv+n*R/m}$$
\begin{equation}
\sympy{Cp} = \sympy{Cv+R/M}
\end{equation}

\section{Equações dos mecanismos de transferências de calor}
\subsection{Condução}
\begin{equation}
Q
\end{equation}
% ---------------------------------------------------------------------------------------
\end{document}
