 % Pacotes e configurações padrão do estilo "article"\
% -------------------------------------
\documentclass[a4paper,11pt]{article} 
% Layout
% --------------------------------------------------------------------------------
%     Gráficos e layout ----------------------------------------------------------------------

\ifx\pdfmatch\undefined
\else
    \usepackage[T1]{fontenc}
    \usepackage[utf8]{inputenc}
\fi
% xetex:
\ifx\XeTeXinterchartoks\undefined
\else
    \usepackage{fontspec}
    \defaultfontfeatures{Ligatures=TeX}
\fi
% luatex:
\ifx\directlua\undefined
\else
    \usepackage{fontspec}
\fi
% End engine-specific settings

%      Fonte --------------------------------------------------------------------------------
%\usepackage{lmodern}
\usepackage{times}
%     Pacotes adicionados -------------------------------------------------------------------
\usepackage{ae}
%     Língua e hifenização ------------------------------------------------------------------
\usepackage[english]{babel}
\usepackage{hyphenat}
%      Outros --------------------------------------------------------------------------------
\usepackage{fancyhdr}
\usepackage{sectsty}
\usepackage{float}
%\usepackage{graphicx}
\usepackage[pdftex]{color,graphicx}
\usepackage{hyperref}
\usepackage{enumerate} % Permite alterar Layout do enumerate
%\usepackage{pdflscape} % Permite alterar a orientação da pagina
%\usepackage{ifthen} % Permite usar condicionais ifelse
%\usepackage[table]{xcolor} % Permite alterar as cores das celulas de uma tabela
\usepackage{amsmath,amssymb} % Ambiente para uso de elementos matemáticos
\usepackage{caption}
\usepackage{subcaption} % permite o uso de multiplas figuras com legenda
%\usepackage{minted} % FOrmatador para códigos de programas
% Layout do documento ------------------------------------------------------------------------
%     Bordas e tamanho da página ------------------------------------------------------------
\usepackage{geometry} 
 \geometry{ % Padrâo ABNT para relatórios
 a4paper,
 left=30mm,
 right=20mm,
 top=30mm,
 bottom=20mm
 }
%     Cabeçalho e Rodapé ---------------------------------------------------------------
\pagestyle{fancy}
  \lhead{}
  \chead{}
  \rhead{}
  \lfoot{}
  \cfoot{}
  \rfoot{\thepage}
%     Númeração ------------------------------------------------------------------------
  \pagenumbering{arabic}
%     Retas do cabeçalho e rodapé ------------------------------------------------------
  \renewcommand{\headrulewidth}{0.5pt}
  \renewcommand{\footrulewidth}{0.5pt}
%     Tamanho da letra de seções e derivadas --------------------------------------------
  \sectionfont{\normalsize}
  \subsectionfont{\small}
%     Hiperlinks ------------------------------------------------------------------------
  \hypersetup{
                  colorlinks,
                  citecolor=black,
                  filecolor=black,
                  linkcolor=black,
                  urlcolor=black
                  }
%     Dados do título e autores --------------------------------------------------------------
%\title{\tituloRelatorio}
\author{Rafael Lima}
%     Definições do pdf ----------------------------------------------------------------------
\hypersetup{
    unicode=false,          % non-Latin characters in Acrobat’s bookmarks
    pdftoolbar=true,        % show Acrobat’s toolbar?
    pdfmenubar=true,        % show Acrobat’s menu?
    pdffitwindow=false,     % window fit to page when opened
    pdfstartview={FitH},    % fits the width of the page to the window    
    pdfauthor={Rafael Lima},     % author
    pdfnewwindow=true      % links in new window
}
%     Outros ----------------------------------------------------------------------------
      %\renewcommand{\thesection}{(\alph{section})} % muda o estilo de númeração das sections
      % alterando a formatação dos numeradores de lista de itens
      \renewcommand\theenumi{\arabic{enumi}}
      \renewcommand\labelenumi{(\textit{\theenumi})}
    \renewcommand\theenumii{\arabic{enumii}}
    \renewcommand\labelenumii{(\textit{\theenumi.\theenumii})}
      
% ---------------------------------------------------------------------------------------


\usepackage{circuitikz}
\usepackage[makestderr]{pythontex}
%\restartpythontexsession{\thesection}

\newcommand{\tituloRelatorio}{Transporte de Calor e Massa\\Lista 1 - Parte 2}
\title{\tituloRelatorio}
\hypersetup{pdftitle={\tituloRelatorio}}% title

% Definições Auxiliares
% -----------------------------------------------------------------
%\input{relat_aux.tex}
\renewcommand{\thesection}{Questão \arabic{section}} 
\renewcommand{\thesubsection}{(\alph{subsection})}
\newcommand{\npy}[1]{\sympy{round(n#1,4)}}
% ----------------------------------~>ø<~---------------------------------------
\begin{document}
% Capa e Índice ---------------------------------------------------------------
\maketitle
% Conteudo -------------------------------------------------------------------
\section{} % q1
\section{} % q2
\begin{sympycode}
# Símbolos algébricos
t = Symbol('t') # Tempo
T = Symbol('T') # Temperatura
x = Symbol('x') # distãncia
L = Symbol('L') # Espessura Placa
A = Symbol('A') # Área da base
Atotal = Symbol('A_total') # Área da base
Aaleta = Symbol('A_aleta') # Área da base
Q = Symbol('Q') # Calor total
h = Symbol('h') # coeficiente conveccao
k = Symbol('k') # Condutividade térmica
n = Symbol('eta')
N = Symbol('N')
T0 = Symbol('T_0') # Temperatura parede interna
Ta = Symbol('T_amb') # Temperatura ambiente
\end{sympycode}

Dado que temos uma aleta para cada região quadrada de $0.6cm^2$, o número total de aletas ao longo da superfície é:
$$
N = A_
$$
\section{} % q3

\begin{sympycode}
# Símbolos algébricos
t = Symbol('t') # Tempo
T = Symbol('T') # Temperatura
x = Symbol('x') # distãncia
L = Symbol('L') # Espessura Placa
A = Symbol('A') # Área da base
Atotal = Symbol('A_total') # Área da base
Aaleta = Symbol('A_aleta') # Área da base
Q = Symbol('Q') # Calor total
h = Symbol('h') # coeficiente conveccao
k = Symbol('k') # Condutividade térmica
n = Symbol('eta')
N = Symbol('N')
T0 = Symbol('T_0') # Temperatura parede interna
Ta = Symbol('T_amb') # Temperatura ambiente

#
dQsa = h*Atotal*(T0 - Ta)
dQa = N*h*Aaleta*n*(T0 - Ta)
dQna = h*(Atotal-Aaleta)*(T0 - Ta)

dQtotal = (dQna+dQa).simplify()
etotal = dQsa/dQtotal

nAtotal = 8*6 # cm^2
nAaleta = 0.2*0.2 # cm^2
nN = 150
nn = 0.65

netotal = etotal.subs([(Atotal,nAtotal),(Aaleta,nAaleta),(N,nN),(n,nn)])

\end{sympycode}

\begin{equation}\label{eq:eglobal}
\epsilon_{global} = \frac{\dot{Q_{sa}}}{\dot{Q_{total}}}
\end{equation}

Temos que o calor transferido pela superfície $\dot{Q_{sa}}$ sem considerar as aletas é

\begin{equation}\label{eq:dq_sa}
\dot{Q_{sa}} = \sympy{dQsa}
\end{equation}

Acrescentando as aletas, temos que o calor transferido pelas aletas $\dot{Q_{a}}$ e o transferido pela superfície restante sem as aletas $\dot{Q_{na}}$ é, respectivamente:

\begin{equation}\label{eq:dq_a}
\dot{Q_{a}} = \sympy{dQa}
\end{equation}

\begin{equation}\label{eq:dq_na}
\dot{Q_{na}} = \sympy{dQna}
\end{equation}

Dado que $\dot{Q_{total}} = \dot{Q_a} + \dot{Q_{na}}$, substituindo as equações \ref{eq:dq_sa}, \ref{eq:dq_na} e \ref{eq:dq_a} na equação \ref{eq:eglobal}:

$$
\epsilon_{global} = \sympy{dQsa/(dQna+dQa)}
$$

Simplificando
\begin{equation}\label{eq:eglobal2}
\epsilon_{global} = \sympy{etotal} = \frac{1}{\sympy{Aaleta/Atotal}(\sympy{N*n -1})+1}
\end{equation}

Subsistituindo os valores de $\sympy{Aaleta} = \npy{Aaleta}$, $\sympy{N} = \npy{N}$ , $\sympy{Atotal} = \npy{Atotal}$ , $\sympy{n} = \npy{n}$:

$$
\epsilon_{global} = \npy{etotal}
$$

% ----------------------------------------------------------------------------
\end{document}
