 % Pacotes e configurações padrão do estilo "article"\
% -------------------------------------
\documentclass[a4paper,11pt]{article} 
% Layout
% --------------------------------------------------------------------------------
\input{relat_layout.tex}
\usepackage{circuitikz}
\usepackage[makestderr]{pythontex}
%\restartpythontexsession{\thesection}

\newcommand{\tituloRelatorio}{Transporte de Calor e Massa\\Lista 1 - Parte 2}
\title{\tituloRelatorio}
\hypersetup{pdftitle={\tituloRelatorio}}% title

% Definições Auxiliares
% -----------------------------------------------------------------
%\input{relat_aux.tex}
\renewcommand{\thesection}{Questão \arabic{section}} 
\renewcommand{\thesubsection}{(\alph{subsection})}
\newcommand{\npy}[1]{\sympy{round(n#1,4)}}
% ----------------------------------~>ø<~---------------------------------------
\begin{document}
% Capa e Índice ---------------------------------------------------------------
\maketitle
% Conteudo -------------------------------------------------------------------
\section{} % q1
\section{} % q2
\begin{sympycode}
# Símbolos algébricos
t = Symbol('t') # Tempo
T = Symbol('T') # Temperatura
x = Symbol('x') # distãncia
L = Symbol('L') # Espessura Placa
A = Symbol('A') # Área da base
Atotal = Symbol('A_total') # Área da base
Aaleta = Symbol('A_aleta') # Área da base
Q = Symbol('Q') # Calor total
h = Symbol('h') # coeficiente conveccao
k = Symbol('k') # Condutividade térmica
n = Symbol('eta')
N = Symbol('N')
T0 = Symbol('T_0') # Temperatura parede interna
Ta = Symbol('T_amb') # Temperatura ambiente
\end{sympycode}

Dado que temos uma aleta para cada região quadrada de $0.6cm^2$, o número total de aletas ao longo da superfície é:
$$
N = A_
$$
\section{} % q3

\begin{sympycode}
# Símbolos algébricos
t = Symbol('t') # Tempo
T = Symbol('T') # Temperatura
x = Symbol('x') # distãncia
L = Symbol('L') # Espessura Placa
A = Symbol('A') # Área da base
Atotal = Symbol('A_total') # Área da base
Aaleta = Symbol('A_aleta') # Área da base
Q = Symbol('Q') # Calor total
h = Symbol('h') # coeficiente conveccao
k = Symbol('k') # Condutividade térmica
n = Symbol('eta')
N = Symbol('N')
T0 = Symbol('T_0') # Temperatura parede interna
Ta = Symbol('T_amb') # Temperatura ambiente

#
dQsa = h*Atotal*(T0 - Ta)
dQa = N*h*Aaleta*n*(T0 - Ta)
dQna = h*(Atotal-Aaleta)*(T0 - Ta)

dQtotal = (dQna+dQa).simplify()
etotal = dQsa/dQtotal

nAtotal = 8*6 # cm^2
nAaleta = 0.2*0.2 # cm^2
nN = 150
nn = 0.65

netotal = etotal.subs([(Atotal,nAtotal),(Aaleta,nAaleta),(N,nN),(n,nn)])

\end{sympycode}

\begin{equation}\label{eq:eglobal}
\epsilon_{global} = \frac{\dot{Q_{sa}}}{\dot{Q_{total}}}
\end{equation}

Temos que o calor transferido pela superfície $\dot{Q_{sa}}$ sem considerar as aletas é

\begin{equation}\label{eq:dq_sa}
\dot{Q_{sa}} = \sympy{dQsa}
\end{equation}

Acrescentando as aletas, temos que o calor transferido pelas aletas $\dot{Q_{a}}$ e o transferido pela superfície restante sem as aletas $\dot{Q_{na}}$ é, respectivamente:

\begin{equation}\label{eq:dq_a}
\dot{Q_{a}} = \sympy{dQa}
\end{equation}

\begin{equation}\label{eq:dq_na}
\dot{Q_{na}} = \sympy{dQna}
\end{equation}

Dado que $\dot{Q_{total}} = \dot{Q_a} + \dot{Q_{na}}$, substituindo as equações \ref{eq:dq_sa}, \ref{eq:dq_na} e \ref{eq:dq_a} na equação \ref{eq:eglobal}:

$$
\epsilon_{global} = \sympy{dQsa/(dQna+dQa)}
$$

Simplificando
\begin{equation}\label{eq:eglobal2}
\epsilon_{global} = \sympy{etotal} = \frac{1}{\sympy{Aaleta/Atotal}(\sympy{N*n -1})+1}
\end{equation}

Subsistituindo os valores de $\sympy{Aaleta} = \npy{Aaleta}$, $\sympy{N} = \npy{N}$ , $\sympy{Atotal} = \npy{Atotal}$ , $\sympy{n} = \npy{n}$:

$$
\epsilon_{global} = \npy{etotal}
$$

% ----------------------------------------------------------------------------
\end{document}
