 % Pacotes e configurações padrão do estilo "article"\
% -------------------------------------
\documentclass[a4paper,11pt]{article}
% Layout
% --------------------------------------------------------------------------------
\input{relat_layout.tex}
\usepackage{circuitikz}
\usepackage[makestderr]{pythontex}
%\restartpythontexsession{\thesection}

\newcommand{\tituloRelatorio}{Transporte de Calor e Massa\\Lista 1}
\title{\tituloRelatorio}
\hypersetup{pdftitle={\tituloRelatorio}}% title

% Definições Auxiliares
% -----------------------------------------------------------------
%\input{relat_aux.tex}
\renewcommand{\thesection}{Questão \arabic{section}}
\renewcommand{\thesubsection}{(\alph{subsection})}
\newcommand{\npy}[1]{\sympy{round(n#1,4)}}
% ----------------------------------~>ø<~---------------------------------------
\begin{document}
% Capa e Índice ---------------------------------------------------------------
\maketitle
% Conteudo -------------------------------------------------------------------
\section{} % q1
\subsection{} %a 
\paragraph{} Transiente pois $\frac{\delta T}{\delta t} \neq 0$

\subsection{} %b
\paragraph{} Unidimensional pois $\frac{\delta^2 T}{\delta y^2} = \frac{\delta^2 T}{\delta z^2} = 0$

\subsection{} %b
\paragraph{} Variável pois $\frac{\delta T}{\delta t} \neq 0$

\section{} % q2 O que  ́e uma condi ̧c a  ̃ o de contorno? Quantas condições de contorno
\paragraph{}Condições de contorno são ... . Sendo necessário especificar apenas 2 condições de contorno para um problema de calor unidimensional.

\section{} % q3 - Panela
% Dùvida: Depois de montar a equação da convecção o que devo fazer?] dado que não tenho a temperatura do meio ou do fundo da panela
% Considera Tf = 100 graus C e Tamb = 25 ?

Assumindo que o calor é distribuído uniformemente ao longo do fundo da panela e tomando a panela com formato de um cilíndro com base circular e com o diâmetro bem maior que a altura, podemos desconsiderar os efeitos da troca de calor das paredes na lateral da panela com a água e modelar, para uma primeira aproximação, o problema de maneira unidimensional, avaliando apenas as trocas de calor ocorridas na direção vertical. Desta forma temos:

\begin{sympycode}
# Symbolic
t = Symbol('t') # Tempo
L = Symbol('L') # Espessura
D = Symbol('D') # Diametro
Qf = Symbol('Q_f') # Energia Gasta Pelo fogao
Q = Symbol('Q') # Energia Recebida pela panela
h = Symbol('h') # coeficiente conveccao
Tamb = Symbol('T_a')
Tf = Symbol('T_f')


a = .85

# Calculo area fundo da panela
A = pi*D*D/4

dQ = h*A*(-Tamb+Tf)
\end{sympycode}

\begin{equation}
 \sympy{Derivative(Q,t)} = \sympy{dQ}
\end{equation}

\begin{sympycode}
#
\end{sympycode}
% q4 - Ferro de Passar Roupas -------------------------------------------------
\section{}

\begin{sympycode}
# Símbolos algébricos
t = Symbol('t') # Tempo
T = Symbol('T') # Temperatura
x = Symbol('x') # distãncia
L = Symbol('L') # Espessura Placa
A = Symbol('A') # Área da base
Q = Symbol('Q') # Energia Fornecida pela Resistẽncia do Ferro
k = Symbol('k') # coeficiente conveccao
Tf = Symbol('T_f') # Temperatura parede externa
T0 = Symbol('T_0') # Temperatura parede interna
h = Symbol('h') # Condutividade térmica

# Valores numéricos
nL = 0.6/100 # m
nA = 160/100/100 # m"2s
ndQ = 800 # W
nk = 60 # W/(m C)
nTf = 112 # graus C
nh = Symbol('h') # Condutividade térmica

dQ = -k*A*Derivative(T,x)
m = (ndQ/(nk*nA))*nL
nT0 = (ndQ/(nk*nA))*nL + nTf
\end{sympycode}

\begin{figure}[H]
\centering
\includegraphics[width = 0.6\linewidth]{./image/lista1/q4}
\end{figure}

\subsection{}

Dado o calor fornedido pela resistência do ferro $\dot{Q}$, as temperaturas na parede interna e na superfície da base do ferro $\sympy{T0}$ e $\sympy{Tf}$, a condutividade térmica no ferro $\sympy{k}$ Pela lei de Fourier da condução de calor temos:

\begin{equation}\label{eq:fourierlaw-cond}
\dot{Q}(x) = \sympy{Derivative(Q,t)} = \sympy{dQ}
\end{equation}

Isolando $\sympy{Derivative(T,x)}$:

$$\frac{\dot{Q}}{\sympy{k*A}} = \sympy{Derivative(T,x)}$$

\subsection{}

Assumindo $\dot{Q}$ constante ao longo de todo o ferro e resolvendo a equação diferencial temos

\begin{equation}\label{eq:q4.Temp}
T = -\left(\frac{\dot{Q}}{\sympy{k*A}}\right)x + \sympy{T0}
\end{equation}

Em $x = 0$, temos $\dot{Q}(0) = \npy{dQ}\ W$, em $x=L$ temos $T(x=L) = \sympy{Tf} = \npy{Tf}^\circ $ e a espessuara da placa $L=\npy{L}$. Logo, a partir da equação \ref{eq:q4.Temp}:

$$T(L) = -\left(\frac{\dot{Q}}{\sympy{k*A}}\right)L + \sympy{T0} = -\frac{\npy{dQ}\cdot \npy{L}}{\npy{k}\cdot \npy{A}} + \sympy{T0}$$
$$T(L) = \sympy{round(-(ndQ*nL)/(nk*nA),4) + T0}$$

Isolando $\sympy{T0}$ e substituindo o valor de $T(L)$:

$$\sympy{T0} = \sympy{round(ndQ/(nk*nA),4)*L + Tf} =  \sympy{round((ndQ/(nk*nA))*nL,4)} + \npy{Tf} \Rightarrow \sympy{T0} = \sympy{round((ndQ/(nk*nA))*nL + nTf,4)}$$

Substituindo o valor de $\sympy{T0}$ na equação \ref{eq:q4.Temp}, encontramos T(x):

\begin{equation}\label{eq:q4.Temp}
T(x) = \sympy{-m*x} + \npy{T0}
\end{equation}

\subsection{}

% Plot Grafico da temperatura

% q5 --------------------------------------------------------------------------
\section{}

Considerando uma espessura $L$ tal que $x=L$ e sabendo se que a temperatura $T(x) = 65x+25$ e que $T(x=L) = 38^\circ C$ temos que $L$ será dado por:

$$L = \frac{T(x=L) - 25}{65} = \frac{(38) - 25}{65} \Rightarrow L = \sympy{round((38-25)/65,3)}$$

% q6 --------------------------------------------------------------------------
\section{}
\begin{sympycode}
# Símbolos algébricos
t = Symbol('t') # Tempo
T = Symbol('T') # Temperatura
x = Symbol('x') # distãncia
L = Symbol('L') # Espessura Placa
A = Symbol('A') # Área da base
Q = Symbol('Q') # Calor total
h = Symbol('h') # coeficiente conveccao
k = Symbol('k') # Condutividade térmica
T0 = Symbol('T_0') # Temperatura parede interna
Ta = Symbol('T_amb') # Temperatura ambiente

# Valores numéricos
nL = 0.2/100 # m
nA = 0.08 # m"2s
nk = 60 # W/(m C)
nh = 15 # W/(m^2 C)
nT0 = 80 # graus C
nTa = 25 # graus C

# Modelo como resistência térmica
Rcond = L/(k*A)
Rconv = 1/(h*A)

# Cálculo por divisor de tensão
Tf = -(Ta-T0)*(Rconv/(Rconv+Rcond))
nTf = Tf.subs([(L,nL),(A,nA),(k,nk),(h,nh),(T0,nT0),(Ta,nTa)])
\end{sympycode}

Considerando $R_{cond} = \sympy{Rcond}$ e $R_{conv} = \sympy{Rconv}$ podemos modelar o sistema como duas resistências em série:

% Modelo aproximado por circuitos

A partir do modelo temos que o fluxo de calor $\dot{Q}$ é

$$\dot{Q} = \frac{\sympy{-Ta+T0}}{R_{cond} + R_{conv}}  = \sympy{-(Ta-T0)/(Rconv+Rcond)}$$

Logo, o valor de $T_f$ pode ser encontrado por $T_f = \dot{Q} R_{conv}$ e portanto:

$$T_f = \sympy{Tf} = \sympy{Tf.simplify()} = \npy{Tf}$$

$$T_f = \frac{\npy{k}(\npy{T0} - \npy{Ta})}{\npy{L}\cdot \npy{h} + \npy{k}}= \npy{Tf}$$

% q7 --------------------------------------------------------------------------
\section{}

\begin{sympycode}
# Símbolos algébricos
t = Symbol('t') # Tempo
T = Symbol('T') # Temperatura
x = Symbol('x') # distãncia
L = Symbol('L') # Espessura Placa
A = Symbol('A') # Área da base
Q = Symbol('Q') # Calor total
h = Symbol('h') # coeficiente conveccao
k = Symbol('k') # Condutividade térmica
T0 = Symbol('T_0') # Temperatura parede interna
Ta = Symbol('T_amb') # Temperatura ambiente
Tf = Symbol('T_f') # Temperatura ambiente
c = Symbol('delta')
E = Symbol('Epsilon')

# Valores numéricos
nL = 0.2/100 # m
nA = 0.08 # m"2s
nk = 60 # W/(m C)
nh = 15 # W/(m^2 C)
nT0 = 80 # graus C
nTa = 25 # graus C

# Modelo como resistência térmica
Rcond = L/(k*A)
Rconv = 1/(h*A)
Rrand = 1/(E*c*A*(Tf*Tf+Ta*Ta)*(Tf+Ta))

Reqv =  Rcond + Rconv*Rrand/((Rconv+Rrand).simplify())

dQ2 = (Ta - Tf)/(Rconv*Rrand/((Rconv+Rrand).simplify()))
dQ1 = (Tf - T0)/Rcond

# Cálculo por divisor de tensão
Tf = -(Ta-T0)*(Rconv/(Rconv+Rcond))
nTf = Tf.subs([(L,nL),(A,nA),(k,nk),(h,nh),(T0,nT0),(Ta,nTa)])
\end{sympycode}

Considerando $R_{cond} = \sympy{Rcond}$ , $R_{conv} = \sympy{Rconv}$ e $R_{rand} = \sympy{Rrand}$ podemos modelar o sistema como duas resistências em série:

% Modelo aproximado por circuitos

A partir do modelo temos que o fluxo de calor $\dot{Q}$ é
%% Tentativa 1
%$$\dot{Q} = \frac{\sympy{-Ta+T0}}{R_{eqv}} =\frac{\sympy{-Ta+T0}}{R_{cond} + \frac{R_{conv}R_{rand}}{R_{conv}+R_{rand}}}$$
%$$\dot{Q} = \sympy{-(Ta-T0)/(Reqv)}$$

%% Tentativa 2
%$$\dot{Q} = \sympy{dQ1} = \sympy{dQ2}$$

???????????
%% Dúvidas Nesta Questão

% q8 --------------------------------------------------------------------------
\section{}

\begin{sympycode}
# Símbolos algébricos
t = Symbol('t') # Tempo
T = Symbol('T') # Temperatura
x = Symbol('x') # distãncia
L = Symbol('L') # Espessura Placa
A = Symbol('A') # Área da base
Q = Symbol('Q') # Calor total
h = Symbol('h') # coeficiente conveccao
k = Symbol('k') # Condutividade térmica
T0 = Symbol('T_0') # Temperatura parede interna
Ta = Symbol('T_amb') # Temperatura ambiente

# Modelo como resistência térmica
Rcond = L/(k*A)
Rconv = 1/(h*A)

# Valores numéricos
nT0 = 300 # graus C
nTa = 100 # graus C
nZ = 8

nL_a = 1*1e-3 #
nA_a = 12*1e-3*nZ #
nk_a = 2 #
nRcond_a = Rcond.subs([(L,nL_a),(k,nk_a),(A,nA_a)])

nL_b = 5*1e-3 #
nA_b = 4*1e-3*nZ #
nk_b = 8 #
nRcond_b = Rcond.subs([(L,nL_b),(k,nk_b),(A,nA_b)])

nL_c = 5*1e-3 #
nA_c = 4*1e-3*nZ #
nk_c = 20 #
nRcond_c = Rcond.subs([(L,nL_c),(k,nk_c),(A,nA_c)])

nL_d = 10*1e-3 #
nA_d = 6*1e-3*nZ #
nk_d = 15 #
nRcond_d = Rcond.subs([(L,nL_d),(k,nk_d),(A,nA_d)])

nL_e = 10*1e-3 #
nA_e = 6*1e-3*nZ #
nk_e = 35 #
nRcond_e = Rcond.subs([(L,nL_e),(k,nk_e),(A,nA_e)])

nL_f = 6*1e-3 #
nA_f = 12*1e-3*nZ #
nk_f = nk_a #
nRcond_f = Rcond.subs([(L,nL_f),(k,nk_f),(A,nA_f)])
\end{sympycode}

\begin{figure}[H]
\centering
\includegraphics[width = 0.6\linewidth]{./image/lista1/q8}
\end{figure}

\subsection{}

Com base na equação \ref{eq:fourierlaw-cond} e a partir dos valores da condutância, espessura e área podemos calcular a resistência térmica para cada uma das partes do bloco. Desta forma obtemos os valores relacionados na tabela \ref{tab:q8.resistence}

\begin{table}[H]
\centering
\begin{tabular}{|cccc|c|}
\hline
 &  $L \ [m]$ &$A\ [m^2]$ & $k \ [W/m ^\circ C]$ & $R_{cond}$\\
\hline
A & \npy{L_a} & \npy{A_a} & \npy{k_a} & \npy{Rcond_a} \\
B & \npy{L_b} & \npy{A_b} & \npy{k_b} & \npy{Rcond_b} \\
C & \npy{L_c} & \npy{A_c} & \npy{k_c} & \npy{Rcond_c} \\
D & \npy{L_d} & \npy{A_d} & \npy{k_d} & \npy{Rcond_d} \\
E & \npy{L_e} & \npy{A_e} & \npy{k_e} & \npy{Rcond_e} \\
F & \npy{L_f} & \npy{A_f} & \npy{k_f} & \npy{Rcond_f} \\
\hline
\end{tabular}
\caption{Valores calculados para resistência térmica}
\label{tab:q8.resistence}
\end{table}

Desta forma podemos avaliar a resistência do bloco como associação das resistências para cada uma das partes conforme a figura \ref{fig:q8.circuit}

\begin{figure}[H]
    \centering
    \begin{circuitikz}[american voltages]
        \draw (0,0) to[R, l=$R_A$,*-*] (2,0);
        \draw (2,0) to (2,1);
        \draw (2,0) to (2,-1);
        \draw (2,1) to[R, l=$R_C$] (4,1);
        \draw (2,0) to[R, l=$R_B$] (4,0);
        \draw (2,-1) to[R, l=$R_C$] (4,-1);
        \draw (4,0) to (4,1);
        \draw (4,0) to (4,-1);
        \draw (4,0) to (5,0);
        \draw (5,0) to (5,0.5);
        \draw (5,0) to (5,-0.5);
        \draw (5,0.5) to[R, l=$R_D$] (7,0.5);
        \draw (5,-0.5) to[R, l=$R_E$] (7,-0.5);
        \draw (7,0) to (7,0.5);
        \draw (7,0) to (7,-0.5);
        \draw (7,0) to (7.5,0);
        \draw (7.5,0) to[R, l=$R_F$,*-*] (9.5,0);
    \end{circuitikz}
    \caption{Circuito Equivalente}
    \label{fig:q8.circuit}
\end{figure}

\begin{sympycode}
#
Ra = Symbol('R_A');
Rb = Symbol('R_B');
Rc = Symbol('R_C');
Rd = Symbol('R_D');
Re = Symbol('R_E');
Rf = Symbol('R_F');

R1 = (Rc/2)*Rb/((Rc/2)+Rb);
nR1 = R1.subs([(Rb,nRcond_b),(Rc,nRcond_c)]);
R2 = (Rd*Re)/(Rd+Re);
nR2 = R2.subs([(Rd,nRcond_d),(Re,nRcond_e)]);

Req1 = Ra+R1+R2+Rf;
nReq1 = Req1.subs([(Ra,nRcond_a),(Rb,nRcond_b),(Rc,nRcond_c),
   (Rd,nRcond_d),(Re,nRcond_e),(Rf,nRcond_f)])

nReq = (nReq1*(800/12))

nQ = (300 -100)/nReq;

nT3 = 300 - nQ*(nRcond_a+nR1)
\end{sympycode}

Definindo as resistências térmicas intermediárias $R_1$ e $R_2$, a partir das camadas formadas pelos materiais $C$, $D$ e $E$ temos:

$$R_1 = \sympy{R1} = \npy{R1}$$
$$R_2 = \sympy{R2} = \npy{R2}$$

Deste modo, podemos definir a resistência térmica do bloco da parede como:

$$R_{eq_1} = R_A + R_1 + R_2 + R_F$$
$$R_{eq_1} = R_A + \sympy{R1} + \sympy{R2} + R_F$$
$$R_{eq_1} = \npy{Req1}$$

Para parede toda a resistência será $R_{eq} = R_{eq_1}\cdot (800cm)/(12cm) = \npy{Req}$. Logo, a partir da resistência térmica podemos calcular a taxa de transferência de calor $\dot{Q}$, para as temperaturas $T_1 = 300^\circ C$ e $T_2 = 100^\circ C$ nas superficies da parede:

$$\dot{Q} = \frac{(T_1 - T_2)}{R_{eq}} = \frac{(300 - 100)}{\npy{Req}} = \npy{Q}\ W$$

\subsection{}

Assumindo a taxa de transferência de calor constante ao longo do bloco, a temperatura no ponto em que as seções B, D e E se encontram pode ser calculada como:
$$
T_3 = T_1 - \dot{Q}(R_A + R_1) $$
$$T_3= 300 - \npy{Q}\cdot ( \npy{Rcond_a} + \npy{R1}) = \npy{T3} ^\circ C
$$

\subsection{}
De maneira similar, a queda de temperatura ao longo do bloco $F$ pode ser calculada como
$$
\Delta T_f =  \dot{Q}\cdot R_F = \npy{Q}\cdot \npy{Rcond_f} = \npy{Q*nRcond_f} ^\circ C
$$

\section{} % q9
\paragraph{}Ao modelar a troca de calor por meio de resistências elétricas, a convecção e a radiação são modelas como resistências em paralelo devido representarem as trocas ocorridas da superfície externa com o meio.

\section{} % q10

\begin{sympycode}
# Símbolos algébricos
t = Symbol('t') # Tempo
T = Symbol('T') # Temperatura
x = Symbol('x') # distãncia
L = Symbol('L') # Espessura Placa
A = Symbol('A') # Área da base
Q = Symbol('Q') # Calor total
h = Symbol('h') # coeficiente conveccao
k = Symbol('k') # Condutividade térmica
T0 = Symbol('T_0') # Temperatura parede interna
Ta = Symbol('T_amb') # Temperatura ambiente

# Modelo como resistência térmica
Rcond = L/(k*A)

# Valores numéricos
nT0 = 300 # graus C
nTa = 100 # graus C
nZ = 8

nL_a = 0.7*12 #
nA_a = 1 #
nk_a = 0.1 #
nRcond_a = Rcond.subs([(L,nL_a),(k,nk_a),(A,nA_a)])

nL_b = 7*12 #
nA_b = 1 #
nk_b = 0.02 #
nRcond_b = Rcond.subs([(L,nL_b),(k,nk_b),(A,nA_b)])

nReq = 2*nRcond_a + nRcond_b
\end{sympycode}

\begin{figure}[H]
\centering
\includegraphics[width = 0.6\linewidth]{./image/lista1/q10}
\end{figure}

Com base na equação \ref{eq:fourierlaw-cond} e a partir dos valores da condutância, espessura e área podemos calcular a resistência térmica para cada uma das partes do bloco. Desta forma obtemos os valores relacionados na tabela \ref{tab:q10.resistence}

\begin{table}[H]
\centering
\begin{tabular}{|cccc|c|}
\hline
 &  $L \ [in]$ &$A $ & $k \ [Btu/h\cdot in ^\circ F]$ & $R_{cond}$\\
\hline
Sheetrock & \npy{L_a} & \npy{A_a} & \npy{k_a} & \npy{Rcond_a} \\
Fiberglass & \npy{L_b} & \npy{A_b} & \npy{k_b} & \npy{Rcond_b} \\
\hline
\end{tabular}
\caption{Valores calculados para resistência térmica}
\label{tab:q10.resistence}
\end{table}

Desta forma podemos avaliar a resistência do bloco como associação das resistências para cada uma das partes conforme a figura \ref{fig:q10.circuit}

\begin{figure}[H]
    \centering
    \begin{circuitikz}[american voltages]
        \draw (0,0) to[R, l=$R_S$,*-*] (2,0);
        \draw (2,0) to[R, l=$R_F$,*-*] (4,0);
        \draw (4,0) to[R, l=$R_S$,*-*] (6,0);
    \end{circuitikz}
    \caption{Circuito Equivalente}
    \label{fig:q10.circuit}
\end{figure}

Deste modo, podemos definir a resistência térmica do bloco da parede como:

$$R_{eq} = R_F + R_S + R_F$$
$$R_{eq} = (\npy{Rcond_a} + \npy{Rcond_b} + \npy{Rcond_a})\ Btu/h ^\circ F$$
$$R_{eq} = \npy{Req}\ Btu/h ^\circ F$$

\section{} % q11

\begin{figure}[H]
\centering
\includegraphics[width = 0.6\linewidth]{./image/lista1/q11}
\end{figure}

\section{} % q12
\begin{sympycode}
# Símbolos algébricos
t = Symbol('t') # Tempo
T = Symbol('T') # Temperatura
x = Symbol('x') # distãncia
L = Symbol('L') # Espessura Placa
A = Symbol('A') # Área da base
Q = Symbol('Q') # Calor total
h = Symbol('h') # coeficiente conveccao
k = Symbol('k') # Condutividade térmica
T0 = Symbol('T_0') # Temperatura parede interna
Ta = Symbol('T_amb') # Temperatura ambiente

# Modelo como resistência térmica
Rcond = L/(k*A)
Rconv = 1/(h*A)

# Valores numéricos
nT0 = 25 # graus C
nTa = 0 # graus C
nh = 18

nL_a = 0.1e-3 #
nA_a = 1.1 #
nk_a = 0.060 #
nRcond_a = Rcond.subs([(L,nL_a),(k,nk_a),(A,nA_a)])

nL_b = 1e-3 #
nA_b = 1.1 #
nk_b = 0.026 #
nRcond_b = Rcond.subs([(L,nL_b),(k,nk_b),(A,nA_b)])

nRconv = Rconv.subs([(A,nA_b),(h,nh)])

nReq1 = 5*nRcond_a + 4*nRcond_b
nReq = nReq1 + nRconv

nQ = (nT0 -nTa)/nReq;
\end{sympycode}

Com base na equação \ref{eq:fourierlaw-cond} e a partir dos valores da condutância, espessura e área podemos calcular a resistência térmica para cada uma das partes do bloco. Desta forma obtemos os valores relacionados na tabela \ref{tab:q10.resistence}

\begin{table}[H]
\centering
\begin{tabular}{|cccc|c|}
\hline
 &  $L \ [m]$ &$A\ [m^2]$ & $k \ [W/m ^\circ C]$ & $R_{cond}$\\
\hline
Fibra & \npy{L_a} & \npy{A_a} & \npy{k_a} & \npy{Rcond_a} \\
Camada de Ar & \npy{L_b} & \npy{A_b} & \npy{k_b} & \npy{Rcond_b} \\
\hline
\end{tabular}
\caption{Valores calculados para resistência térmica}
\label{tab:q10.resistence}
\end{table}

A resistência de condução será dado por

$$R_{conv} = \sympy{Rconv} = \npy{Rconv}$$

Desta forma podemos avaliar a resistência da jaqueta como associação das resistências térmica da fibra $R_f$ e da camada de ar $R_a$ somado da resitência de convecção:

$$R_{eq} = 5\cdot R_{f} 4\cdot R_{a} + R_{conv}$$
$$R_{eq} = 5\cdot \npy{Rcond_a} + 4\cdot \npy{Rcond_b} + \npy{Rconv}$$
$$R_{eq} = \npy{Req}$$

Logo, a partir da resistência térmica, podemos calcular a taxa de transferência de calor $\dot{Q}$, para as temperaturas $T_1 = \npy{T0}^\circ C$ e $T_a = \npy{Ta}^\circ C$ :

$$\dot{Q} = \frac{(T_1 - T_a)}{R_{eq}} = \frac{(\npy{T0} - \npy{Ta})\ ^\circ C}{\npy{Req}\ ^\circ C/W} = \npy{Q}\ W$$

Logo a resposta é \textbf{(c)} $126\ W$

\section{} % q13

\begin{figure}[H]
\centering
\includegraphics[width = 0.6\linewidth]{./image/lista1/q13}
\end{figure}

\begin{sympycode}
# Símbolos algébricos
t = Symbol('t') # Tempo
T = Symbol('T') # Temperatura
x = Symbol('x') # distãncia
L = Symbol('L') # Espessura Placa
A = Symbol('A') # Área da base
Q = Symbol('Q') # Calor total
h = Symbol('h') # coeficiente conveccao
k = Symbol('k') # Condutividade térmica
T0 = Symbol('T_0') # Temperatura parede interna
Ta = Symbol('T_amb') # Temperatura ambiente

Rc1 = Symbol('R_C1');
Re1 = Symbol('R_E1');

# Modelo como resistência térmica
Rcond = L/(k*A)

# Valores numéricos
ndT = 50 # graus C

# Ao longo do comprimento
nL_c1 = 10e-3 #
nA_c1 = 1*2e-6 #
nk_c = 400 #
nRcond_c1 = Rcond.subs([(L,nL_c1),(k,nk_c),(A,nA_c1)])

nL_e1 = 10e-3 #
nA_e1 = 1*2e-6 #
nk_e = 0.4 #
nRcond_e1 = Rcond.subs([(L,nL_e1),(k,nk_e),(A,nA_e1)])

Req1 = (Rc1*Re1)/(Rc1+Re1);
nReq1 = (nRcond_e1*nRcond_c1)/(nRcond_e1+nRcond_c1)
nQ1 = ndT/nReq1;

# Da esquerda para direita
nL_c2 = 1e-3 #
nA_c2 = 10*2e-6 #
nRcond_c2 = Rcond.subs([(L,nL_c1),(k,nk_c),(A,nA_c1)])

nL_e2 = 1e-3 #
nA_e2 = 10*2e-6 #
nRcond_e2 = Rcond.subs([(L,nL_e2),(k,nk_e),(A,nA_e2)])

nReq2 = nRcond_e2 + nRcond_c2
nQ2 = ndT/nReq2;

# De cima para baixo
nL_c3 = 2e-3 #
nA_c3 = 10*1e-6 #
nRcond_c3 = Rcond.subs([(L,nL_c3),(k,nk_c),(A,nA_c3)])

nL_e3 = 2e-3 #
nA_e3 = 10*1e-6 #
nRcond_e3 = Rcond.subs([(L,nL_e3),(k,nk_e),(A,nA_e3)])

nReq3 = nRcond_e3 + nRcond_c3
nQ3 = ndT/nReq3;
\end{sympycode}

Com base na equação \ref{eq:fourierlaw-cond} e a partir dos valores da condutância, espessura e área podemos calcular a resistência térmica para cada uma das partes do bloco. Desta forma obtemos os valores relacionados na tabela \ref{tab:q10.resistence}

\begin{table}[H]
\centering
\begin{tabular}{|cccc|c|}
\hline
 &  $L \ [m]$ &$A\ [cm^2]$ & $k \ [W/m ^\circ C]$ & $R_{cond} [^\circ C\ W]$\\
\hline
Cobre & \npy{L_c1} & \npy{A_c1*1e6} & \npy{k_c} & \npy{Rcond_c1} \\
Epoxy & \npy{L_e1} & \npy{A_e1*1e6} & \npy{k_e} & \npy{Rcond_e2} \\
\hline
Cobre & \npy{L_c2} & \npy{A_c2*1e6} & \npy{k_c} & \npy{Rcond_c2} \\
Epoxy & \npy{L_e2} & \npy{A_e2*1e6} & \npy{k_e} & \npy{Rcond_e2} \\
\hline
Cobre & \npy{L_c3} & \npy{A_c3*1e6} & \npy{k_c} & \npy{Rcond_c3} \\
Epoxy & \npy{L_e3} & \npy{A_e3*1e6} & \npy{k_e} & \npy{Rcond_e3} \\
\hline
\end{tabular}
\caption{Valores calculados para resistência térmica em cada sentido}
\label{tab:q10.resistence}
\end{table}

\subsection{Taxa de transferência ao longo da barra}

Desta forma podemos avaliar a resistência térmica da barra ao longo de seu comprimento como

$$R_{eq1} = \sympy{Req1}$$
$$R_{eq1} = \frac{\npy{Rcond_c1}\cdot \npy{Rcond_e1}}{\npy{Rcond_c1}+\npy{Rcond_e1}} $$
$$R_{eq1} = \npy{Req1}\ ^\circ C\ W$$

E portanto a taxa de transferência de calor será $Q_1 = \Delta T/R_{eq1} = \npy{Q1}\ W$

\subsection{Taxa de transferência de calor da esquerda para direita}

Desta forma podemos avaliar a resistência térmica da barra ao longo de seu comprimento como

$$R_{eq2} = R_{c2} + R_{e2}$$
$$R_{eq2} = \npy{Rcond_c2} + \npy{Rcond_e2}$$
$$R_{eq2} = \npy{Req2}\ ^\circ C\ W$$

E portanto a taxa de transferência de calor será $Q_1 = \Delta T/R_{eq2} = \npy{Q2}\ W$

\subsection{Taxa de transferência de calor de cima para baixo}

Desta forma podemos avaliar a resistência térmica da barra ao longo de seu comprimento como

$$R_{eq3} = R_{c3} + R_{e3}$$
$$R_{eq3} = \npy{Rcond_c3} + \npy{Rcond_e3}$$
$$R_{eq3} = \npy{Req3}\ ^\circ C\ W$$

E portanto a taxa de transferência de calor será $Q_1 = \Delta T/R_{eq3} = \npy{Q3}\ W$


\section{} % q14
\begin{sympycode}
# Símbolos algébricos
t = Symbol('t') # Tempo
T = Symbol('T') # Temperatura
x = Symbol('x') # distãncia
L = Symbol('L') # Espessura Placa
A = Symbol('A') # Área da base
Q = Symbol('Q') # Calor total
h = Symbol('h') # coeficiente conveccao
k = Symbol('k') # Condutividade térmica
T0 = Symbol('T_0') # Temperatura parede interna
Ta = Symbol('T_amb') # Temperatura ambiente

#

\end{sympycode}


\section{} % q15

\begin{figure}[H]
\centering
\includegraphics[width = 0.6\linewidth]{./image/lista1/q15}
\end{figure}

A taxa de transferência de calor do corpo 1 para o 2 é dado por

\begin{equation}\label{eq:q1.q12}
\dot{Q_{1\to 2}} = \frac{\sigma T_1^4 + \sigma T_2^4}{\frac{1-E_1}{A_1 E_1} + \frac{1}{A_1 F_{12}} + \frac{1-E_2}{A_2 E_2}}
\end{equation}

Aonde $A_1$ e $A_2$ é obtido como $A_1 = 4\pi r_1^2 = 4\pi (0.3)^2 = 1.15$ e $A_2 = 4\pi r_2^2 = 4\pi (1.2)^2 = 4.52$. E $E_1 = 0.9$ e $E_1 = 0.5$. O fator de forma de 1 para 2 é

$$
F_{12} = 0.5\left\{ 1 - \left[1 + \left(\frac{R_2}{h}\right)^2 \right]^{-0.5}\right\} = 0.2764
$$

Podemos relacionar o fator de forma de 1 para 2 com o fator de forma de 2 para 1 por $A_1F_{12} = A_2F_{21}$, a partir do qual temos que $F_{21} = 0.0691$. Assim,substituindo os valores na equação \ref{eq:q1.q12}, temos:
$$
\dot{Q_{1\to 2}} = 8554 W
$$
% ----------------------------------------------------------------------------
\end{document}
