%     Gráficos e layout ----------------------------------------------------------------------

\ifx\pdfmatch\undefined
\else
    \usepackage[T1]{fontenc}
    \usepackage[utf8]{inputenc}
\fi
% xetex:
\ifx\XeTeXinterchartoks\undefined
\else
    \usepackage{fontspec}
    \defaultfontfeatures{Ligatures=TeX}
\fi
% luatex:
\ifx\directlua\undefined
\else
    \usepackage{fontspec}
\fi
% End engine-specific settings

%      Fonte --------------------------------------------------------------------------------
%\usepackage{lmodern}
\usepackage{times}
%     Pacotes adicionados -------------------------------------------------------------------
\usepackage{ae}
%     Língua e hifenização ------------------------------------------------------------------
\usepackage[english]{babel}
\usepackage{hyphenat}
%      Outros --------------------------------------------------------------------------------
\usepackage{fancyhdr}
\usepackage{sectsty}
\usepackage{float}
%\usepackage{graphicx}
\usepackage[pdftex]{color,graphicx}
\usepackage{hyperref}
\usepackage{enumerate} % Permite alterar Layout do enumerate
%\usepackage{pdflscape} % Permite alterar a orientação da pagina
%\usepackage{ifthen} % Permite usar condicionais ifelse
%\usepackage[table]{xcolor} % Permite alterar as cores das celulas de uma tabela
\usepackage{amsmath,amssymb} % Ambiente para uso de elementos matemáticos
\usepackage{caption}
\usepackage{subcaption} % permite o uso de multiplas figuras com legenda
%\usepackage{minted} % FOrmatador para códigos de programas
% Layout do documento ------------------------------------------------------------------------
%     Bordas e tamanho da página ------------------------------------------------------------
\usepackage{geometry} 
 \geometry{ % Padrâo ABNT para relatórios
 a4paper,
 left=30mm,
 right=20mm,
 top=30mm,
 bottom=20mm
 }
%     Cabeçalho e Rodapé ---------------------------------------------------------------
\pagestyle{fancy}
  \lhead{}
  \chead{}
  \rhead{}
  \lfoot{}
  \cfoot{}
  \rfoot{\thepage}
%     Númeração ------------------------------------------------------------------------
  \pagenumbering{arabic}
%     Retas do cabeçalho e rodapé ------------------------------------------------------
  \renewcommand{\headrulewidth}{0.5pt}
  \renewcommand{\footrulewidth}{0.5pt}
%     Tamanho da letra de seções e derivadas --------------------------------------------
  \sectionfont{\normalsize}
  \subsectionfont{\small}
%     Hiperlinks ------------------------------------------------------------------------
  \hypersetup{
                  colorlinks,
                  citecolor=black,
                  filecolor=black,
                  linkcolor=black,
                  urlcolor=black
                  }
%     Dados do título e autores --------------------------------------------------------------
%\title{\tituloRelatorio}
\author{Rafael Lima}
%     Definições do pdf ----------------------------------------------------------------------
\hypersetup{
    unicode=false,          % non-Latin characters in Acrobat’s bookmarks
    pdftoolbar=true,        % show Acrobat’s toolbar?
    pdfmenubar=true,        % show Acrobat’s menu?
    pdffitwindow=false,     % window fit to page when opened
    pdfstartview={FitH},    % fits the width of the page to the window    
    pdfauthor={Rafael Lima},     % author
    pdfnewwindow=true      % links in new window
}
%     Outros ----------------------------------------------------------------------------
      %\renewcommand{\thesection}{(\alph{section})} % muda o estilo de númeração das sections
      % alterando a formatação dos numeradores de lista de itens
      \renewcommand\theenumi{\arabic{enumi}}
      \renewcommand\labelenumi{(\textit{\theenumi})}
    \renewcommand\theenumii{\arabic{enumii}}
    \renewcommand\labelenumii{(\textit{\theenumi.\theenumii})}
      
% ---------------------------------------------------------------------------------------

